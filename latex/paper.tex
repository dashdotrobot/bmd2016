\documentclass{bmd2016p}

\begin{document}

\begin{center}
\fontsize{14}{20}{\bf Buckling and dynamic collapse of the bicycle wheel}
\end{center}

%%%%%%%%%%%%%%%% authors %%%%%%%%%%%%%%%
\begin{center}
\normalsize{\bf{M. Ford$^{*}$, J.M. Papadopoulos$^\#$, 
            O. Balogun$^\dag$}}
\end{center} 

\begin{center}
\begin{tabular}{c}
$^*$ Department of Mechanical Engineering\\
Northwestern University\\
2145 Sheridan Road, Evanston, IL 60208, USA\\
e-mail: mford@u.northwestern.edu\\[2.5ex]

$^\#$ Department of Mechanical and Industrial Engineering\\
Northeastern University\\
360 Huntington Ave., Boston, MA 02115, USA\\
e-mail: j.papadopoulos@neu.edu\\[2.5ex]

$^\dag$ Department of Civil Engineering\\
Northwestern University\\
2145 Sheridan Road, Evanston, IL 60208, USA\\
e-mail: o-balogun@northwestern.edu\\
\end{tabular}
\end{center}

\section*{ABSTRACT}

...

\begin{keywords}
bicycle wheel, 
buckling, 
bifurcation.
\end{keywords}





\section{INTRODUCTION}

The bicycle wheel is a prestressed structure and is susceptible to buckling under internal forces. As the spokes are tightened uniformly, the rim deforms radially to accommodate the spoke strain. At a critical tension, the system reaches a bifurcation point and the rim buckles out of its initial plane. The post-buckling configuration is generally stable and the original shape of the wheel can be recovered by reducing the tension. Buckling can also be triggered by external forces in an otherwise stable wheel leading to a release of strain energy. This is an unstable process which generally leads to catastrophic failure.

Despite its implications for wheel strength, the buckling problem has never received a rigorous treatment to our knowledge. Jobst Brandt alludes to buckling in his practical manual for wheelbuilders~\cite{Brandt1993c}:

\begin{quotation}
\noindent``If the wheel becomes untrue in two large waves during stress relieving, the maximum, safe tension has been exceeded. Approach this tension carefully to avoid major rim distortions. When the wheel loses alignment from stress relieving, loosen all spokes a half turn before retruing the wheel.''
\end{quotation}

He did not discuss the problem further, but suggested that wheel failure commonly occurs due to a loss of lateral stability caused by spoke buckling. Pippard and Francis~\cite{Pippard1932d} derived a model for lateral stiffness based on the elastic foundation model, but did not discuss stability and neglected any effects of spoke tension.

Flexural-torsional buckling of rings can be treated as a special case of buckling of arches, where the included angle is allowed to go to $2\pi$. Timoshenko and Gere~\cite{Timoshenko1961a} gave a formula for the critical load for a ring with doubly-symmetric cross-section subjected to radial loads. The theory of flexural-torsional buckling of monosymmetric arches (bicycle rims have only one plane of symmetry) was broadly formalized by Trahair and Papangelis~\cite{Trahair1987b} using the virtual work approach to derive the equilibrium and stability equations. Their theory has been extended to treat arches restrained by continuous~\cite{Pi2002b} or discrete~\cite{Bradford2002d} elastic supports and elastic end restraints~\cite{Guo2014b}.

The problem of the prestressed bicycle wheel is unique for a number of reasons. First, the buckling load is internal to the structure. Second, the spokes act both as elastic restraints resisting buckling and as prestressing elements causing buckling. Third, the lateral, radial, and torsional restraining actions of the spokes are coupled---i.e. lateral deflection at a spoke may produce lateral, radial, and torsional reactions. These considerations extend to other structural systems. Large observation wheels such as the London Eye~\cite{Mann2001a} and the Singapore Flyer~\cite{Allsop2009a} resemble large bicycle wheels and achieve lateral stability due to the bracing angle of prestressed cables, and must be designed against flexural-torsional buckling.

Here we derive a general theory for buckling of a spoked bicycle wheel under self-tension and investigate several special cases. Furthermore, we analyze large deformation post-buckling behavior and show how local buckling of individual spokes can lead to collapse of an otherwise sub-critical wheel.





\section{DEFORMATION OF THE WHEEL}
\subsection{Rim equations}

The rim is modeled as an initially circular beam with a constant cross-section. For simplicity we will assume that the shear center and centroid of the cross-section coincide. Except for warping deformation, plane cross-sections remain planar and shear flexibility of the cross-section is neglected. Furthermore, we assume that the rim radius is much larger than the height of the rim profile.

The in-plane curvature $\kappa_1$ is assumed to remain constant during flexural-torsional buckling. The in-plane curvature, out-of-plane curvature and twist rate are
\begin{equation}\label{eq:k1}
\kappa_1 = 1/R
\end{equation}
\begin{equation}\label{eq:k2}
\kappa_2 = u'' - \frac{\phi}{R}
\end{equation}
\begin{equation}\label{eq:k3}
\kappa_3 = \phi' + \frac{1}{R} u'
\end{equation}
The symbol $()'$ indicates a derivative with respect to $s$. It can easily be verified that Equations \ref{eq:k2} and \ref{eq:k3} simplify to the standard equations for straight beams in the limit $R\rightarrow \infty$. The curvature and twist rate produce a bending moment and twisting moment equal to
\begin{equation}\label{eq:M2}
M_2 = EI_{11} \kappa_2
\end{equation}
\begin{equation}\label{eq:M3}
M_3 = GJ \kappa_3 + EI_w \kappa_3''
\end{equation}
The total strain energy is decomposed into lateral bending, uniform torsion, and warping terms which are related to the curvatures.
\begin{equation}\label{eq:Urim}
U_{rim} = \int_0^{2\pi} \left( \frac{1}{2} EI_{22} \kappa_2^2 + \frac{1}{2} GJ \kappa_3^2 + \frac{1}{2} EI_w (\kappa_3')^2 \right)\, ds
\end{equation}
After buckling, a differential line segment $ds$ deforms to $dS$. The apparent shortening of the line segment along the $s$ direction is then
\begin{equation}\label{eq:ds}
ds - dS = ds - \sqrt{ds^2 + \left(\frac{du}{ds}ds\right)^2} \approx \frac{1}{2} (u')^2 \, ds
\end{equation}
The axial compressive force in the rim moves through this differential displacement, performing virtual work equal to
\begin{equation}\label{eq:Vrimr}
V_{rim} = \frac{1}{2} \int_0^{2\pi} N_r (u)^2 \, ds
\end{equation}


\subsection{Spoke equations}

In this paper we consider bicycle wheels with slender prestressed spokes. We model a single spoke as an elastic bar pinned at each end. This ensures that the spoke force is always collinear with the spoke axis. We fix our global coordinate system to the hub, which is assumed to be rigid. Therefore, the displacement of the hub node of a spoke is zero. The linearized elongation of a spoke is given by
\begin{equation}\label{eq:selong}
\Delta l = \bm{u}_n\cdot \hat{\bm{n}}_1
\end{equation}
where $\bm{u}_n$ is the displacement vector of the spoke nipple and $\hat{\bm{n}}_1$ is a unit vector pointing from the spoke nipple to the hub eyelet. The linearized rotation of the spoke is given by
\begin{equation}\label{eq:srot1}
\Omega_1 = \frac{1}{l} \bm{u}_n\cdot \hat{\bm{n}}_2
\end{equation}
\begin{equation}\label{eq:srot2}
\Omega_2 = \frac{1}{l} \bm{u}_n\cdot \hat{\bm{n}}_3
\end{equation}
where $\hat{\bm{n}}_2$ and $\hat{\bm{n}}_3$ are two orthogonal unit vectors orthogonal to $\hat{\bm{n}}_1$, and $l$ is the spoke length.

The elongation produces a net force on the rim parallel to the original spoke axis equal to
\begin{equation}\label{eq:sF1}
f_1 = \left(\frac{E_sA_s\Delta l}{l}\hat{\bm{n}}_1\right) = \frac{E_sA_s}{l} \hat{\bm{n}}_1 \cdot \bm{u}_n 
\end{equation}
The rotation of the spoke produces net forces on the rim perpendicular to the original spoke axis due to the rotation of the initial tension vector.
\begin{equation}\label{eq:sF2}
f_2 = T \sin{\Omega_1} \approx \frac{T}{l} \hat{\bm{n}}_2 \cdot \bm{u}_n
\end{equation}
\begin{equation}\label{eq:sF3}
f_3 = T \sin{\Omega_2} \approx \frac{T}{l} \hat{\bm{n}}_3 \cdot \bm{u}_n
\end{equation}
since $\hat{\bm{n}}_1,\hat{\bm{n}}_2,\hat{\bm{n}}_3$ are mutually orthogonal, the total force on the rim is given by $\bm{f} = \bm{k}_f \cdot \bm{u}_n$, where the spoke stiffness matrix is given by
\begin{equation}\label{eq:kf}
\bm{k}_f = \frac{E_sA_s}{l}\hat{\bm{n}}_1\hat{\bm{n}}_1 + \frac{T}{l}(\hat{\bm{n}}_2\hat{\bm{n}}_2 + \hat{\bm{n}}_3\hat{\bm{n}}_3)
\end{equation}
The first term in Eqn.~\ref{eq:kf} is the {\bf elastic stiffness} and the second term is the {\bf tension stiffness}. The tension stiffness is the component responsible for the transverse vibrations of a guitar string, e.g.

Following the approach of Smith~\cite{Smith1901a} and Pippard~\cite{Pippard1932d}, we will approximate the spoke system as a continuum which exerts a distributed load on the rim. The stiffness per unit length is defined by the sum of the spoke stiffnesses in one periodic grouping of spokes, divided by the arc length of the grouping.
\begin{equation}\label{eq:kbar}
\bar{\bm{k}} = \frac{1}{2\pi R}\left(\frac{n_s}{n_p}\right) \sum_i^{n_p} \bm{k}_{f, i}
\end{equation}


\subsection{Total potential energy and buckling criterion}
The buckled shape of the wheel must be periodic, so we assume a buckled shape of the form
\begin{equation}\label{eq:modeshape}
\begin{split}
u(s) &= \delta u_n \cos{\frac{ns}{R}} \\
\phi(s) &= \delta\phi_n \cos{\frac{ns}{R}}
\end{split}
\end{equation}
The wheel becomes unstable when the increase in strain energy is exactly balanced by the virtual work for a perturbing displacement $\delta u_n, \delta\phi_n$, i.e. when
\begin{equation}\label{eq:TotPot}
U_{rim} - V_{rim} = 0
\end{equation}
Substituting the buckling mode shape Eqn. \ref{eq:modeshape} into Eqns. \ref{eq:Urim} and \ref{eq:Vrimr} and substituting into Eqn. \ref{eq:TotPot} yields the critical buckling criterion for the $n$th mode.
\begin{equation}\label{eq:Tcrit}
...
\end{equation}
\section{CONCLUSIONS}

...

\bibliographystyle{naturemag}
\bibliography{C:/Users/matt/OneDrive/Documents/Research/bibtex/Papers-BMD2016.bib}

\end{document}
